
% !TEX encoding = UTF-8 Unicode

% This is a simple template for a LaTeX document using the "article" class.
% See "book", "report", "letter" for other types of document.

\documentclass[11pt]{article} % use larger type; default would be 10pt

\usepackage[utf8]{inputenc} % set input encoding (not needed with XeLaTeX)

%%% Examples of Article customizations
% These packages are optional, depending whether you want the features they provide.
% See the LaTeX Companion or other references for full information.

%%% PAGE DIMENSIONS
\usepackage{geometry} % to change the page dimensions
\geometry{a4paper} % or letterpaper (US) or a5paper or....
% \geometry{margin=2in} % for example, change the margins to 2 inches all round
% \geometry{landscape} % set up the page for landscape
%   read geometry.pdf for detailed page layout information

\usepackage{graphicx} % support the \includegraphics command and options

% \usepackage[parfill]{parskip} % Activate to begin paragraphs with an empty line rather than an indent

%%% PACKAGES
\usepackage{booktabs} % for much better looking tables
\usepackage{array} % for better arrays (eg matrices) in maths
%\usepackage{paralist} % very flexible & customisable lists (eg. enumerate/itemize, etc.)
\usepackage{verbatim} % adds environment for commenting out blocks of text & for better verbatim
\usepackage{subfig} % make it possible to include more than one captioned figure/table in a single float
% These packages are all incorporated in the memoir class to one degree or another...

%%% HEADERS & FOOTERS
\usepackage{fancyhdr} % This should be set AFTER setting up the page geometry
\pagestyle{fancy} % options: empty , plain , fancy
\renewcommand{\headrulewidth}{0pt} % customise the layout...
\lhead{}\chead{}\rhead{}
\lfoot{}\cfoot{\thepage}\rfoot{}

%%% SECTION TITLE APPEARANCE
\usepackage{sectsty}
\allsectionsfont{\sffamily\mdseries\upshape} % (See the fntguide.pdf for font help)
% (This matches ConTeXt defaults)

%%% ToC (table of contents) APPEARANCE
\usepackage[nottoc,notlof,notlot]{tocbibind} % Put the bibliography in the ToC
\usepackage[titles,subfigure]{tocloft} % Alter the style of the Table of Contents
\renewcommand{\cftsecfont}{\rmfamily\mdseries\upshape}
\renewcommand{\cftsecpagefont}{\rmfamily\mdseries\upshape} % No bold!

%%% END Article customizations

\usepackage[spanish]{babel}
\usepackage{listings} 
%%% The "real" document content comes below...

\title{Casos de Usos - Buscaminas }

%\date{} % Activate to display a given date or no date (if empty),
         % otherwise the current date is printed 

\begin{document}

\begin{titlepage}

\begin{center}
\vspace*{-1in}
\begin{figure}[htb]
\begin{center}
\includegraphics[width=8cm]{./espol}
\end{center}
\end{figure}

FACULTAD DE INGENIERIA \\
\vspace*{0.15in}
ELECTRICA Y COMPUTACION \\
\vspace*{0.6in}
\begin{large}
ING: JAVIER TIBAU\\
\end{large}
\vspace*{0.4in}
\begin{Large}
\textbf{PROYECTO BUSCAMINAS EN ANDROID} \\
\end{Large}
\vspace*{0.5in}
\begin{large}
Integrantes:\\Leonel Ramirez\\José Vélez\\Kevin Campuzano\\ 
\end{large}
\vspace*{0.3in}
\rule{80mm}{0.1mm}\\
\vspace*{0.1in}
\begin{large}
Casos de uso - Diagrmas de Clases
\end{large}
\end{center}

\end{titlepage}






\maketitle
%\tableofcontents % No hace falta un TOC en un artículo corto


\chapter{Lista de Casos de Uso}

\begin{enumerate}
	\item {Escoger opción del menú juego}
	\item {Escoger opción reglas}	
	\item{Escoger opción desarrolladores}
	\item{Escoger opción puntaje}
\end{enumerate}

\section{Caso}Escoger  opción del menú juego.\newline\newline
Descripcion: En este caso de uso se escogerá las opciones para inicializar el juego y se verán  las diferentes  puntuaciones y las reglas del mismo.

	\subsection{Escenario}
	 Escoger la opción incorrecta que no va corresponder con los requerimientos del usuario.\newline \newline
Actor: jugador.\newline
Acciones: Ingresar a la opción juego.\newline
Resultado: no permite inicializar el juego.\newline

\subsection{Escenario}
Escoger la opción correcta que permita iniciar el  juego, ingreso exitoso.\newline\newline
Actor: jugador.\newline
Asunciones:  Se asume  que la opción escogido fue la correcta.\newline
Resultado: Se iniciara la ventana con las opciones del juego exitosamente.
 
\subsection{Escenario}
 
Escoger el nivel..\newline\newline
Actor: jugador
Asunciones: Vamos asumir que se escogió el nivel correctamente.\newline
Acciones: El actor dará clic en el botono siguiente.\newline
Resultado: Presentara la nueva ventana con el juego cargado.\newline

\subsection{Escenario}
Jugar buscaminas\newline \newline
Actor: jugador\newline
Asunciones: Vamos asumir que el actor ha ingresado correctamente al juego.\newline
Acciones: Presiona una casilla aleatoriamente.\newline
Resultado: el cronometro empieza a correr entonces empieza el juego.\newline

\subsection{Escenario}
Reiniciar.\newline \newline
Actor: Jugador. \newline
Asunciones: Se asume que se dio clic en el botón correcto.\newline
Acciones: El  juego vuelve a cargarse a partir de cero.\newline
Resultado: Presenta una tablero nuevo con todos las minas ocultas.\newline

\subsection{Escenario}
Salir.\newline \newline
Actor: jugador.\newline
Asunciones: Se asume que se dio clic en el botón  correcto.\newline
Acciones: Destruye toda las actividades del juego.\newline
Resultado: Salir exitosamente  de la consola del juego.\newline
\newpage


\section{Caso} Escoger opción reglas\newline\newline
Descripción: en esta opción se muestran todas las reglas del juego

\subsection{Escenario}
Se presionara la opción regla que es del juego.\newline \newline
Actor: Jugador.\newline
Asunciones: Se asume que ha entrado a la opción regla con éxito.\newline
Acciones: Se muestran todas las reglas del juego.\newline
Resultado: El jugador puede leer las reglas del juego.\newline

\section{Caso}
Escoger opción puntaje.\newline \newline
Descripción: Se conoce quienes tienes el tiempo en todos los juegos creados.
\subsection{Escenario}
Se presiona la opción puntaje.\newline \newline
Actor: Jugador.\newline
Asunciones: Se asume que el usuario entro a la opción con éxito.\newline
Resultado: Se muestran los nombres con los mejores puntajes.\newline

\end{document}